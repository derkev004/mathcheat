%For documentclass cheatsht see https://github.com/derkev004/cheatsht
\documentclass[draft]{cheatsht}
%Compile with lualatex, use package luatex85 with luatex 0.85 or higher
\usepackage{luatex85}
%Needed packages / commands for this specific cheat sheet
\usepackage{amsmath,amssymb,cancel,tikz,hologo}
\usepackage{lipsum}
\newcommand{\Tikz}{{Ti\textcolor{orange}{\emph{k}}Z}}
\newcommand{\notx}{\textbf{not} }

%pdf meta data
\pdfinfo{
  /Title (mathcheat.pdf)
  /Creator (TeX)
  /Producer (LuaLaTeX 0.95.0)
  /Author (Konrad)
  /Subject (Math / TikZ / PGF Cheat Sheet)
  /Keywords (lualatex, latex, lua, tex, math, tikz, pgf)
}

\begin{document}
%Yes, title works in place and it can be reused
\title{\LaTeX~Math Cheat Sheet}

\section{Packages}

\begin{tabular}{@{}ll@{}}
\texttt{amsmath} & Use math macros \\
\texttt{amssymb} & Use more math symbols \\
\texttt{cancel} & Cross out text
\end{tabular}

Use before \verb!\begin{document}!.
Usage: \verb!\usepackage{!\textit{package name}\verb!}!

\section{Math Mode}
\subsection{Inline Math}
Don't use \verb!$!\,\verb!…!\,\verb!$! with \LaTeX.
Use \verb!\(! \verb!…!\,\verb!\)! instead.

\subsection{Displayed Math}
Don't use \verb!$$!\,\verb!…!\,\verb!$$! with \LaTeX. Use
\verb!\[! \verb!…!\,\verb!\]! instead. \\

If you use \AmS\hspace{1pt}math, don't use \verb!\[! \verb!…!\,\verb!\]! either,
use \verb!\begin{displaymath} !\ldots\verb! \end{displaymath}! (unnumbered) and
\verb!\begin{equation} !\ldots\verb! \end{equation}! (numbered).

\subsection{Plain Text in Math Mode}
Use \verb!\text{!\,\verb!…!\,\verb!}! or \verb!\textnormal{!\,\verb!…!\,\verb!}!
or \verb!\mathrm{!\,\verb!…!\,\verb!}! for inline text.
Note the different outcomes.\\
Examples:
\( math mode, \text{text,~} \textnormal{normaltext,~} \mathrm{mathrm}\)\\
Use \verb!\intertext{!\,\verb!…!\,\verb!}! for a complete line, only in
displayed mode.

\section{Sets of Equations}

\begin{tabular}{@{}ll@{}}
\texttt{\&=} & Typeset and aligns equations on =. Works with any relation.\\
& Use \verb!\mathrel{!\,\verb!…!\,\verb!}! or
\verb!\stackrel{!\,\emph{top}\,\verb!}{!\,\emph{bot}\,\verb!}!
  for custom relations\\
\texttt{\&} & Add another column\quad
  \begin{tabular}{ll}
    \texttt{\textbackslash\textbackslash} & Add another line
  \end{tabular}
\end{tabular}

\subsection{align}
Note that align must \notx be set in math mode!\\
Usage: \verb@\begin{align} aa &< A & b &\stackrel{!}{=} B \\@
\verb!c &\mathrel{=_{42}} C & d &= D \end{align}!
Outcome: \begin{align} aa &< A & b &\stackrel{!}{=} B \\
c &\mathrel{=_{42}} C & d &= D \end{align}\vspace{-14pt}
\subsection{aligned}
Allows for further mathstuff left/right, must be set in math mode.
Usage: \verb!\begin{aligned} aa &= A & b &= B! \ldots~\verb!\end{aligned}!
Outcome: \begin{displaymath}
  \begin{aligned} aa &= A & b &=B \\ c &= C & d &= D \end{aligned}
\end{displaymath}

\subsection{gather}
Centered equations, one column. Must \notx be set in math mode!\\
Usage: \verb!\begin{gather} aa = A \\ b = B \end{gather}!\\
Outcome: \begin{gather} aa = A \\ b = B \end{gather}

\section{Long Terms/Equations}
\subsection{multline}
Set long terms with multiple lines. Must \notx be set in math mode!
Usage: \verb!\begin{multline} A = 1 +! \ldots~\verb!+ 5 \\+ 6 + 7 +! \ldots \\
\verb!+ 14 + 15 \end{multline}!\\
Outcome:
\begin{multline} A = 1 + 2 + 3 + 4 + 5 \\
  + 6 + 7 + 8 + 9 + 10 + 11 + 12 + 13 + 14 + 15
\end{multline}

\subsection{split}
Set long equations with multiple lines. Must be set in math mode. \\
Usage: \verb!\begin{split} A &= 5+9+3 \\ &= 14+3 \\ A &= 17 \end{split}!
Outcome: \begin{displaymath}
  \begin{split} A &= 5+9+3 \\ &= 14+3 \\ A &= 17 \end{split}
\end{displaymath}

\section{Cases}
Set if-then-else cases. Must be set in math mode.\\
Usage: \verb!\begin{cases} 1 & \text{if A=...} \\!\\
\verb!2 & \text{if B=...} \end{cases}!\\
Outcome:\begin{displaymath}\begin{cases}
1&\text{if A=...}\\
2&\text{if B=...}
\end{cases}\end{displaymath}

\section{Matrices}
\subsection{matrix}
Set simple matrices. Must be set in math mode.\\
\begin{tabular}{@{}ll@{}}
  \verb! \( \begin{bmatrix} a & b \\ c & d \end{bmatrix} \)!
    & \hspace{-4pt}\(\begin{bmatrix} a & b \\ c & c \end{bmatrix}\) \vspace{4pt}
\end{tabular}
\begin{tabular}{@{}llllll@{}}
  \verb!Bmatrix! & \(\begin{Bmatrix} a & b \\ c & c \end{Bmatrix}\) &
  \verb!vmatrix! & \(\begin{vmatrix} a & b \\ c & c \end{vmatrix}\) &
  \verb!Vmatrix! & \(\begin{Vmatrix} a & b \\ c & c \end{Vmatrix}\)
    \vspace{4pt}\\
  \verb!pmatrix! & \(\begin{pmatrix} a & b \\ c & c \end{pmatrix}\) &
  \verb!matrix! & \(\begin{matrix} a & b \\ c & c \end{matrix}\) \vspace{4pt}
\end{tabular}

A \verb!smallmatrix! for inline use only is available as well.
\(\begin{smallmatrix}a & b \\ c & d \end{smallmatrix}\)

\subsection{array}
Set flexible matrices. Allows for further mathstuff left/right,
must be set in math mode.\\
Usage: \verb!\begin{array}{lc|r} a & b & c \\!
\verb!\hline d & e & f \end{array}!\\
\verb!l! for left aligned, \verb!c! for centered,
\verb!r! for right aligned column.
\verb!|! for optional vertical line. \verb!\hline! adds a horizontal line.\\
Outcome:
\begin{displaymath}
  \begin{array}{lc|r} a & b & c \\ \hline d & e & f \end{array}
\end{displaymath}

\section{Fractions}
\subsection{frac}
Usage: \verb! \( \frac{1}{2} \)!\\
Outcome: \(\frac{1}{2}\)

\subsection{cfrac}
Set continued fractions, must be set in math mode.\\
Usage: \verb!\cfrac{1}{1 + \cfrac{2}{33} }!\\
Outcome: \begin{displaymath}
  \cfrac{1}{1 + \cfrac{2}{33} }
\end{displaymath}

\section{Roots}
Usage: \verb! \( \sqrt[3]{8} \)!\hspace{1.5cm}Outcome: \(\sqrt[3]{8}\)\\
If the root looks like this \(\sqrt[a_3]{b}\), use \verb!\leftroot{n}! and
\verb!\uproot{n}! to correct positioning.\\
Usage: \verb! \( \sqrt[\uproot{3}\leftroot{1} a_3]{8} \)! \\
Outcome: \(\sqrt[\uproot{3}\leftroot{1} a_3]{8}\)


\section{Miscellaneous}
\subsection{Numbering}    %\tag{linear}\label{eq:linear}
Use \verb!align*!, \verb!gather*!, \verb!multline*! to supress numbering.\\
Use \verb!\nonumber! to supress numbering for current line in
any math environment.

\subsection{Brackets}
Use \verb!\leftX! paired with \verb!\rightY! with X and Y being \verb!(!
\verb!)! \verb![! \verb!]!\\
\begin{tabular}{@{}llll@{}}
  \verb!\langle! for \(\langle\) & \verb!\rangle! for \(\rangle\) &
  \verb!\lbrace! for \(\lbrace\) & \verb!\rbrace! for  \(\rbrace\) \\
  \verb!\lfloor! for \(\lfloor\) & \verb!\lceil! for \(\lceil\) &
  \verb!\vert! for \(\vert\) & \verb!\Vert! for \(\Vert\)
\end{tabular}

or \verb!.! to supress one bracket. These brackets adapt in height to fit their
inner object.\\
Usage: \verb!\( \left( \frac{1}{2} \right) \)!\\
Outcome: \( \left( \frac{1}{2} \right) \) as opposed to \((\frac{1}{2})\)

\subsection{Multi-line limits, Custom Operators \& Sidesets}
\begin{tabular}{@{}ll@{}}
Usage: \verb!\sum_{ \substack{1<i<p \\ 1<j<q} }! &
  Outcome: \(\sum_{\substack{1<i<p\\1<j<q}}\) \\
Usage: \verb!\operatorname{mysin}{x}! & Outcome: \(\operatorname{mysin}{x}\)\\
Usage: \verb!\sideset{_{1}^{2}}{_{3}^{4}}\sum!&
  Outcome: \(\sideset{_{1}^{2}}{_{3}^{4}}\sum\)
\end{tabular}

\subsection{cancel}
Usage: \verb!\cancel{22}!\hspace{1.5cm}Outcome: \(\cancel{22}\)\vspace{4pt}\\
\begin{tabular}{@{}lll@{}}
  \verb!\cancel!  \(\frac{(x+2)\cancel{(x-1)}}{\cancel{(x-1)}(x+1)}\) &
  \verb!\bcancel! \(\frac{(x+2)\bcancel{(x-1)}}{\bcancel{(x-1)}(x+1)}\) &
  \verb!\xcancel! \(\frac{(x+2)\xcancel{(x-1)}}{\xcancel{(x-1)}(x+1)}\)
\end{tabular}

%\subsection{Multi line limits}
\subsection{Sub-/Superscription}
Use \verb!_{n}! to subscript and \verb!^{n}! to superscript \verb!n!.\\
Usage: \verb!\( a_{1_{1}}^{2} \)!\hspace{1.5cm}Outcome: \( a_{1_{1}}^{2}\)

\subsection{Symbols}
\begin{tabular}{@{}ll||ll@{}}
  \(\sum_{i=1}^{n}\) & \verb!\sum_{i=1}^{n}! & \(\prod_{i=1}^{n}\) &
    \verb!\prod_{i=1}^{n}! \\
  \(\rightarrow\) & \verb!\rightarrow! & \(\leftarrow\) & \verb!\leftarrow! \\
  \(\Rightarrow\) & \verb!\Rightarrow! & \(\Leftarrow\) & \verb!\Leftarrow! \\
  \(\uparrow\) & \verb!\uparrow! & \(\downarrow\) & \verb!\downarrow! \\
  \(\uparrow\) & \verb!\uparrow! & \(\downarrow\) & \verb!\downarrow! \\
  \(\xrightarrow[3]{44}\) & \verb!\xrightarrow[3]{44}!
    & \(\xleftarrow[3]{44}\) & \verb!\xleftarrow[3]{44}! \\
  \(\pi\) & \verb!\pi! & \(\aleph\) & \verb!\aleph! \\
  \(\overrightarrow{abc}\) & \verb!\overrightarrow{abc}!
    & \(\overleftarrow{abc}\) & \verb!\overleftarrow{abc}!\\
  \(\widehat{abc}\) & \verb!\widehat{abc}! & \(\widetilde{abc}\) &
    \verb!\widetilde{abc}!\\
  \(\overbrace{abc}\) & \verb!\overbrace{abc}! & \(\underbrace{abc}\) &
    \verb!\underbrace{abc}!\vspace{-3pt}\\
  \(\ast\) & \verb!\ast! & \(\cdot\) & \verb!\cdot!\\
  \(\times\) & \verb!\times! & \(\div\) & \verb!\div!\\
  \(\leq\) \(\nleq\) & \verb!\leq \nleq!
    & \(\geq\) \(\ngeq\) & \verb!\geq \ngeq!\\
  \(\nless\) \(\ngtr\) & \verb!\nless \ngtr! & \(\neq\) & \verb!\neq!\\
  \(\pm\) & \verb!\pm! & \(\sim\) & \verb!\sim!\\
  \(\in\) & \verb!\in! & \(\notin\) & \verb!\notin!\\
  \(\forall\) & \verb!\forall! & \(\exists\) & \verb!\exists!\\
  \(\sin(x)\) & \verb!\sin(x)! & \(\cos(x)\) & \verb!\cos(x)!\\
  \(\log n\) & \verb!\log n! & \(\ln n\) & \verb!\ln n!\\
\end{tabular}
\signature{
  2016/6/25 K. Konrad. This work has been released into the public domain\\
  by the copyright holder. This applies worldwide.
}

%Add a new cheat sheet with newcheat
%-------------------------------------------------------------------------------
\newcheat
%-------------------------------------------------------------------------------
\title{\Tikz~\&~PGF~Cheat Sheet}


\section{Package \& Basic Usage}

\begin{tabular}{@{}ll@{}}
\texttt{tikz} & Create Drawings for use in \hologo{(La)TeX},
  or \hologo{ConTeXt}
\end{tabular}

Use before \verb!\begin{document}!.
Usage: \verb!\usepackage{!\textit{package name}\verb!}!\\[2pt]
Everything \Tikz-related must be preceded by a \verb!\tikz! command or placed
inside \verb!\begin{tikzpicture} !\ldots\verb! \end{tikzpicture}!.

\section{Geometric figures, grid panes, axes}

\subsection{Rectangle}
\verb!\tikz \draw (0,0) rectangle (0.2,0.2);!
\tikz \draw (0,0) rectangle (0.2,0.2);\\
\verb!\draw! draws the given figure starting from the first coordinate, ending
with the second coordinate, the standard unit is cm. Don't forget the semicolon!

\subsection{Circle/Ellipses}
\verb!\tikz \draw (1mm,1mm) circle [radius=0.1cm];!
\tikz \draw (1mm,1mm) circle [radius=0.1cm];\\
\verb![radius=0.15cm]! is an optional parameter, the default radius is 0pt. The
coordinate determines the center of the circle.\\[1pt]
\begin{minipage}{7cm}
  \verb!\tikz \draw (3mm,3mm) circle!\\
  \verb![x radius=0.3cm, y radius=0.15cm];!
\end{minipage}
\tikz \draw (3mm,3mm) circle [x radius=0.3cm,y radius=0.15cm];\\
Use two radii for ellipses.
\subsection{Line}
\begin{minipage}{7cm}
  \verb!\tikz \draw (0,0) -- (1,0.2) -- (1.5,0.1);!
  Lines are drawn coordinates connected by \verb!--!.
\end{minipage}
\tikz \draw (0,0) -- (1,0.2);\\[1pt]

\subsection{Dot}
\verb!\tikz \filldraw (0,0) circle [radius=2pt];!
\tikz \filldraw (0,0) circle [radius=2pt];\\
Use \verb!filldraw! for any filled figure.
% </foldme>
\subsection{Triangles/Polygons}
\verb!\tikz \draw (0,0) -- (0.5,0.2) -- (1,0.1) -- cycle;!
\tikz \draw (0,0) -- (0.5,0.2) -- (1,0.1) -- cycle;\\
\verb!\tikz \draw (0,0)--(.4,0.2)--(.8,0.1)--(.4,0)--cycle;!
\tikz \draw (0,0)--(.4,0.2)--(.8,0.1)--(.4,0)--cycle;\\[1pt]
Add \verb!cycle! after connected lines to generate a polygon.

\subsection{Arcs}
\begin{minipage}{7cm}
  \verb!\tikz \draw (0,0) arc!\\
  \verb![start angle=0, end angle=120, radius=4mm];!
\end{minipage}
\tikz \draw (0,0) arc [start angle=0, end angle=120, radius=4mm];\\[1pt]
\begin{minipage}{7cm}
  \verb!\tikz \draw (0,0) arc [start angle=0!\\
  \verb!, end angle=120, x radius=8mm, y radius=4mm];!
\end{minipage}
\tikz \draw (0,0) arc
  [start angle=0, end angle=120,x radius=8mm, y radius=4mm];\\

~\\[3cm]
\lipsum[1-10]


\end{document}
